The increase in the use of GPS devices on cars, smartphones, and remote sensors permitted the collection of information about the spatial distribution of entities,
%
as this data is collected constantly, the time information of measurements is an essential aspect of it.
%
\textit{"Spatio-temporal"} refers to the data with both spatial and temporal information, and it can be of different types, according to~\cite{ansari2020spatiotemporal}: events, geo-referenced data items, geo-referenced time series, moving objects, and trajectories.
%
Events are data with a spatial position and an interval of existence; one example is seismic activity. 
%
Geo-referenced data items and time series are very similar and consist of measurements of attributes in specific positions and determined timestamps. 
%
In time series, it is considered the complete history of measurements. Examples are the measurement of temperature in weather stations.
%
Again, moving objects and trajectories are very similar, consisting of data of objects identified that move through space as time pass, with the information of the intermediary positions.
%
A typical example is the routes of vehicles.

%
With the many types of spatio-temporal data, data mining techniques are used to obtain the essential characteristics to analyze~\cite{ansari2020spatiotemporal}, and one of the most common is clustering techniques.
%
By creating clusters, the data is separated into groups that relate to real events in the observed system. It is used in many domains such as human mobility, healthcare, seismology, and climate science~\cite{BELHADI2020103857}.
%
This clustering can transform different types of spatio-temporal data into events by considering each cluster an event that occurs in a region on space (that contains all points from the cluster) and has a period of existence (from the earliest point to the last).
%

While these techniques help analysts reveal possible patterns, these clusters' spatio-temporal nature and complexity make their visual representation challenging. 
%
The results will have at least three dimensions, one temporal and two spatial, and may include categories and labels requiring techniques to view multivariate data. 
%
It is not desired to visualize only two dimensions of the principal three, and a simple dimensionality reduction from 3D to 2D is also not desired because it will not be possible to answer questions with separated time and space, for example, \textit{"where did it happen?"} and \textit{"when it happened?"}.

%
On the other side, widely used geographic map-based techniques suffer from over-plotting. 
%
Space-time Cube is a 3D cube where the space is represented in its base and the time on the height. However, as it is represented on a 2D screen, a projection is necessary, and it can distort space and time.
%
Also, the use of animations is very common. Still, it is not possible to visualize states of different timestamps simultaneously and to make comparisons is necessary to keep track of the data, dealing with a cognitive limitation of memorization.
%
In another technique, small-multiples use sub-figures to compose a plot; each sub-figure represent a fixed time interval, and for that reason, can ignore the representation of time, but the size of the final plot limits the number of sub-figures.
%

To overcome these limitations, Events-Vis was developed, a method for visualizing spatio-temporal events that seek to represent this three-dimensional information in a static temporal plot.
%
Focusing on preserving areas and neighborhoods of events, we visually approximate the 2D spatial context into a 1D axis by using dimensionality reduction and optimization techniques. 
%
Each event from the data is represented as a rectangle; the width is used to represent the temporal duration and the height to represent the spatial area.

The organization of this document is: on Section~\ref{ch:background} is presented the fundamentals necessary on this project, and on Section~\ref{ch:related-works} are presented already done work in spatio-temporal visualization, on Section~\ref{ch:methodology} is presented the development of this work, the details of the created visualization technique, the implementation of a web visualization tool and the description are how the method was evaluated, on the last Section~\ref{ch:results} is shown the results obtained from the visualization, presenting the evaluation of the method and a use case on real-world data.

The implementation of the proposed method, with the web interfaces, the data, and the experiments are available publicly at~\cite{Giovani2021}.