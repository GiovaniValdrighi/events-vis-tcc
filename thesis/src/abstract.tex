The analysis of time-evolving clusters is an essential task in the study of spatio-temporal data.
People often use these clusters to represent events automatically detected in many fields, such as human mobility and disease outbreaks.
%
Performing visual analysis of this data type is challenging for current state-of-the-art techniques due to factors such as spatial span covered by clusters, spatiotemporal intersections, and temporal evolution.
%
All of this makes widely used geographical map-based techniques suffer from overplotting and cluttering, therefore, ineffective for this purpose.
%
Visualization techniques used to analyze results use animation or interactivity to represent the three dimensions, but they show limitations on interpretation.
%
To overcome these limitations, we present Events-Vis, a method for visualizing spatio-temporal clusters event data in a static temporal plot by representing the space in one dimension. We linearize the space using two strategies: a greedy algorithm and a convex optimization. In both cases, our goal is to preserve neighborhoods and intersections. 
%
We demonstrate the effectiveness of our method in a series of experiments and a case study using both synthetic and real-world datasets.